\section{Metodologias}
Nesta seção, é descrita a metodologia utilizada para o desenvolvimento do  protótipo de gerenciamento de projetos colaborativos.

O projeto se baseou nos conceitos e práticas ágeis oriundas do framework Scrum, as quais permitiram o acompanhamento das atividades de forma mais dinâmica por ciclos curtos de desenvolvimento, segundo (Stair e Reynolds, 2015), o desenvolvimento de software é o processo que vai da sua concepção à disponibilização para uso. Envolve a identificação do problema ou oportunidade, entendimento de suas regras, programação e implantação. Além disso, foram definidas etapas para esse projeto, conforme descrito a seguir:

\subsection{Levantamento de funcionalidades}
Nesta etapa, a equipe realizou reuniões com os membros do projeto para discutir o conteúdo e definir as funcionalidades do protótipo, utilizando as práticas de Lean Inception que segundo Caroli (2018), tem como objetivo alinhar a equipe em relação ao produto que será construído.
Após várias reuniões online e presencial, conseguimos identificar quais funcionalidades poderiam haver no protótipo e principalmente quais tecnologias empregar para o desenvolvimento do projeto, foi identificado a necessidade de utilizar funcionalidades objetivas e uma interface de bom entendimento, quanto às tecnologias, que fossem atuais, permitindo a fácil manutenibilidade e confiabilidade.

\subsection{As tecnologias definidas para o desenvolvimento}
Segundo You (2014), \textbf{Vue.js} é um framework progressivo Javascript que permite a criação de aplicações Web reativas através de uma linguagem única utilizando Single File Components e tecnologia Client Side Rendering, foi usado para criar as telas da aplicação, processar formulários diversos, consumir a API da aplicação e tudo relacionado a construção da UI/UX, um framework simples de implementar e utilizar em aplicações Web, por sua linguagem ser Javascript, é extremamente leve, flexível e extensível. 
% Abaixo alguns pacotes utilizados para estender funcionalidades
% \begin{itemize}
% 	\item Vue Router: Pacote oficial do Vue responsável por gerenciar as páginas e rotas do cliente, simulando uma aplicação com várias páginas em um único arquivo utilizando Javascript. 
% 	\item Axios, um cliente simples baseado em promises para fazer requisições HTTP (GET, POST, PUT, DELETE, etc).
% 	\item Pinia é gerenciador de estados globais para Vue, utilizado para gerenciar os estados de login do usuário de forma local.
% 	\item Vue3 Markdown It é um renderizador de markdown em tempo real para Vue.js. Usado para melhorar o sistema de visualização de documentos com o poderoso Markdown. 
% \end{itemize}

Segundo a OpenJS Foundation (2019), \textbf{Node.js} é uma plataforma e interpretador de código Javascript que funciona fora do navegador, foi utilizado para criar os serviços backend da aplicação, como as respostas da API REST, comunicação com o banco de dados e tudo relacionado a infraestrutura da aplicação, é uma tecnologia simples de utilizar, devido o mesmo fazer o uso da linguagem Javascript para rodar as tarefas. É possível instalar pacotes utilizando o gerenciador de pacotes NPM para acelerar o desenvolvimento.
% Abaixo a lista de pacotes
% \begin{itemize}
% 	\item	Express.js, é um micro framework Web backend que agiliza o processo de construção da API, sendo extremamente extensível e flexível.
% 	\item Prisma é um ORM que permite a comunicação com o banco de dados através do código.
% 	\item Express Session permitiu criar as sessões de usuário, para gerenciar logins e logouts. 
% 	\item Bcrypt permite encriptar e decriptar as senhas.
% 	\item Typescript é typeset do Javascript, que permite uma programação mais segura usando tipos de dados, POO melhorada e vários outros recursos.
% 	\item CORS permite a transmissão de dados entre origens cruzadas, permitindo a comunicação entre a app frontend e a API.
% \end{itemize}

Segundo MariaDB Foundation (2009), \textbf{MariaDB} é um sistema de gerenciamento de banco de dados relacional (RDBMS) de código aberto que se originou como um "fork" do MySQL,  foi utilizado para gerenciar o banco de dados do usuário final.
