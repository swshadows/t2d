\section{Introdução}


Quando se trata de projetos desenvolvidos em equipe, gerenciá-los de maneira produtiva é um desafio significativo. Especialmente para empresas de tecnologia envolvidas no desenvolvimento de \textit{software}, lidar com múltiplos projetos requer o uso de ferramentas adequadas para garantir a entrega de qualidade aos clientes dentro dos prazos estabelecidos. Como afirmou (Caroli, 2018), “mesmo em um projeto ágil, é essencial alinhar e definir objetivos, estratégias e escopo do produto antes de começar a executá-lo”. Um projeto de desenvolvimento de \textit{software} envolve a solução de problemas por meio de aplicações, \textit{frameworks}, sistemas e tecnologias, resultando em produtos como \textit{websites}, aplicativo \textit{desktop} e \textit{mobile}, \textit{bots}, códigos de automação, entre outros, que resolvem parcial ou totalmente um determinado problema. Nesse contexto, existem várias etapas envolvidas no projeto de \textit{software}, com o objetivo de criar um produto de qualidade tanto para os usuários finais que o utilizarão, clientes que financiarão o trabalho, além dos próprios desenvolvedores responsáveis pela codificação e manutenção do projeto.

% Ao adotar ferramentas eficientes de gerenciamento de projetos, como metodologias ágeis e plataformas de colaboração, é possível melhorar a produtividade da equipe, aumentar a transparência, controlar o escopo e garantir a entrega de um produto de alta qualidade dentro dos prazos pré-estabelecidos.

Quando projetos não são gerenciados com metodologias ágeis ou sem o uso adequado de ferramentas, podem ocorrer várias falhas e desafios que afetam a eficiência e o sucesso do projeto. Segundo Humble (2014) essas são as possíveis falhas identificadas em um início de projeto:
\begin{itemize}
	\item \textbf{Falta de visibilidade e controle}: Sem ferramentas adequadas, é difícil ter uma visão clara do progresso do projeto, das tarefas atribuídas a cada membro da equipe e dos prazos. Isso pode levar a uma falta de controle sobre o projeto como um todo, resultando em atrasos e riscos não identificados.
	\item \textbf{Comunicação ineficiente}: Sem uma ferramenta de comunicação centralizada, a comunicação entre os membros da equipe pode se tornar confusa e desorganizada. As informações podem se perder em e-mails, mensagens dispersas ou até mesmo em conversas pessoais. Isso pode levar a mal-entendidos, retrabalho e decisões baseadas em informações desatualizadas.
	\item \textbf{Dificuldade na colaboração}: Sem ferramentas adequadas para colaboração, as equipes podem enfrentar desafios ao trabalhar em conjunto. A falta de compartilhamento de arquivos, controle de versões e capacidade de edição colaborativa pode prejudicar a produtividade e resultar em problemas de integração de trabalho.
	\item \textbf{Gestão de alterações ineficiente}: Sem uma ferramenta para rastrear alterações e solicitações, pode ser difícil gerenciar e controlar as mudanças no escopo do projeto. Isso pode levar a um escopo em constante mudança e falta de controle sobre o trabalho adicional que está sendo realizado.
	\item \textbf{Falta de transparência}: Sem ferramentas que forneçam visibilidade para todas as partes interessadas envolvidas no projeto, pode haver uma falta de transparência sobre o andamento, problemas e resultados alcançados. Isso pode levar a expectativas mal alinhadas e falta de confiança entre a equipe e os stakeholders.
\end{itemize}


% No contexto em questão, surgiu a necessidade de pesquisar e desenvolver um protótipo funcional para a gestão colaborativa de projetos. Esse protótipo abrange diversas funcionalidades, tanto para o usuário como para o gerenciamento dos projetos. No que diz respeito ao usuário, é possível realizar o cadastro, login e logout, além de atualizar informações como e-mail, nome de usuário e senha, e até mesmo deletar a conta. Já no âmbito do gerenciamento dos projetos, as funcionalidades incluem a criação de novos projetos, a edição do nome e descrição de um projeto existente, a exclusão de projetos, e a exibição dos projetos associados ao usuário.

O protótipo também oferece outras funcionalidades, como a criação de novos documentos, a obtenção de todos os documentos de um projeto, a visualização dos documentos compartilhados com o usuário logado, o compartilhamento de documentos com outros usuários, a remoção do compartilhamento de um documento, a atualização do nome do documento, a exclusão de documentos, a obtenção de informações de um documento e o salvamento de documentos.