\section{Considerações Finais}
O artigo enfatizou a importância das metodologias ágeis e das ferramentas apropriadas para o gerenciamento de projetos de desenvolvimento de software. Destaca-se que a falta de visibilidade e controle, comunicação ineficiente, dificuldade na colaboração, gestão de alterações ineficientes e falta de transparência são problemas comuns em projetos mal gerenciados.

O artigo atingiu o objetivo que era  desenvolvimento de um protótipo funcional para a gestão colaborativa de projetos. Esse protótipo realizou de forma satisfatória através de testes para realizar o cadastro, login, criação e compartilhamento de projetos e documentos.

Essa solução contribui para uma melhor coordenação e execução dos projetos dentro dos prazos estabelecidos.
