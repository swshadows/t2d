\par\noindent\rule{\textwidth}{0.4pt}
\par Resumo: Com a evolução da tecnologia em todo o mundo da informática, a automação nas empresas, o crescimento acelerado na busca incansável por melhores produtos, tem se intensificado a procura por ferramentas que utilizam metodologias ágeis que melhor se encaixe para as empresas de desenvolvimento de software utilizarem, transparência, análise e adaptação explicam como esses pilares são essenciais para alcançar resultados significativos em termos de produtividade (Sutherland, 2014). Atualmente, os micros e pequenos empreendimentos têm dificuldades para trabalhar com gerenciamento de tarefas de forma mais simplificada, sendo necessário recorrer a ferramentas privadas para poder realizar o gerenciamento correto dos seus projetos.  Esse projeto utilizou como metodologia para o desenvolvido do protótipo, os conceitos básicos do framework das metodologias ágeis, realizou uma pesquisa bibliográfica em livros e artigos especializados na área em questão e para a implementação através de Typescript para geração do (\textit{backend, frontend}), Vue.js para o \textit{frontend} e o MariaDB para criação do banco de dados. O projeto buscou uma solução para facilitar a integração entre equipes de forma colaborativa trazendo transparência, inspeção e adaptação no tempo certo, fazendo com que as atividades possam ser gerenciadas de forma presencial ou remota permitindo a interação constante entre os membros envolvidos nos projetos.

\noindent\textit{Palavras-chaves}: Desenvolvimento; Gerenciamento; Projeto; Protótipo; Produtividade.

\par\noindent\rule{\textwidth}{0.4pt}